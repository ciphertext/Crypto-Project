\documentclass[12pt]{amsart}
\usepackage{geometry} 
\usepackage{calc}
\usepackage{eucal}
\usepackage{cite}

\pagestyle{plain}
\geometry{letterpaper}

\usepackage[retainorgcmds]{IEEEtrantools}
\usepackage{graphicx}
\usepackage{color}
\usepackage{textcomp}
\usepackage{pstricks}
\usepackage{pdftricks}
\begin{psinputs}
	\usepackage{pstricks}
	\usepackage{color}
	\usepackage{pstcol}
	\usepackage{pst-plot}
	\usepackage{pst-tree}
	\usepackage{pst-eps}
	\usepackage{multido}
	\usepackage{pst-node}
	\usepackage{pst-eps}
\end{psinputs}

\usepackage[linesnumbered,vlined,ruled]{algorithm2e}
%%

\newcommand{\mmod}{\hspace{-6pt}\pmod}
\newcommand{\defeq}{\stackrel{\text{\tiny def}}{=}}
\newcommand{\defmark}{$\mathbf{Definition.}$ }
\newcommand{\exmark}{$\mathbf{Ex.}$ }
\newcommand{\forcevspace}[1][8pt]{$ $\\\vspace{#1}}

\newcommand{\adv}[2][]{\mathbf{Adv}^{#1}_{#2}}
\renewcommand{\exp}[2][]{\mathbf{Exp}^{#1}_{#2}}
\newcommand{\pr}[1][]{\mathbf{Pr}[#1]}
\newcommand{\C}{\mathcal{C}}
\renewcommand{\P}{\mathcal{P}}
\newcommand{\K}{\mathcal{K}}
\newcommand{\E}{\mathcal{E}}
\newcommand{\X}{\mathcal{X}}
\newcommand{\Y}{\mathcal{Y}}

\newcommand{\Zz}{\mathbb{Z}}

\setlength{\parindent}{0pt}
\setlength{\parskip}{1ex plus 0.5ex minus 0.2ex}

\newtheorem{theorem}{Theorem}[section]
\newtheorem{lemma}[theorem]{Lemma}

\title{Homomorphic Encryption}
\author{Justin Sahs}
%%% BEGIN DOCUMENT
\begin{document}
\maketitle
\section{Correctness of the Somewhat Homomorphic Scheme}

\defmark A somewhat homomorphic scheme is said to be \emph{correct} if $D_\E(E_\E(m)) = m$, and for $c_i = E_\E(m_i)$, $D_\E(V_\E(C,\langle c_0, \ldots, c_n \rangle)) = C(\langle m_0,\ldots,m_n\rangle)$.

\begin{lemma}\label{ecorrect}
If $c$ is output from $E_\E(m)$, then $c = a\cdot p + (2b + m)$ where $|2b+m| < p$.
\end{lemma}
\begin{proof}
From Lemma A.1, $c = a\cdot p + (2b + m)$ for some $a$ and $b$ such that $|2b+m| \le \tau 2^{\rho+3}$. Then,

\begin{IEEEeqnarray*}{r;l}
	|2b+m| &\le \tau 2^{\rho+3}\\
		   &= \gamma \omega(\log\lambda) 2^{\rho+3}\\
		   &= \gamma \omega(\log\lambda) 2^{\omega(\log\lambda)}\\
		   &= \omega(\eta^2\log\lambda) \omega(\log\lambda) 2^{\omega(\log\lambda)}\\
		   &= \omega(\rho\Theta(\lambda\log^2\lambda)\log\lambda) \omega(\log\lambda) 2^{\omega(\log\lambda)}\\
		   &= \omega(\omega(\log\lambda)\Theta(\lambda\log^2\lambda)\log\lambda) \omega(\log\lambda) 2^{\omega(\log\lambda)}\\
		   &= \omega(\lambda\log^5\lambda 2^{\log\lambda})\\
\noalign{Additionally,}\\
	p &= \omega(2^\eta)\\
	  &= \omega(2^{\rho\Theta(\lambda\log^2\lambda)})\\
	  &= \omega(2^{\omega(\log\lambda)\Theta(\lambda\log^2\lambda)})\\
	  &= \omega(2^{\lambda\log^3\lambda})\\
\end{IEEEeqnarray*}
so we have
$$2^{\log\lambda} \le |2b+m| \le 2^{\log^2\lambda}$$
so $|2b+m| < 2^{\lambda\log^3\lambda} \le p$.
\end{proof}

\begin{theorem}
$\E$ is correct.
\end{theorem}
\begin{proof}
From Lemma ~\ref{ecorrect} and Lemma A.2, we have that
\begin{IEEEeqnarray*}{r;l}
m' &\leftarrow (c\bmod p) \bmod 2\\
	&= 2b+m \bmod 2\\
	&= m \bmod 2\\
	&= m
\end{IEEEeqnarray*}
for any $c = E_\E(m)$ or $c = V_\E(C,\langle c_0,\ldots,c_n\rangle)$, so the scheme is correct.
\end{proof}

\section{Correctness of the Fully Homomorphic Scheme}



\bibliography{citations}{}
\bibliographystyle{plain}

\end{document}